\documentclass{scrartcl}
\usepackage[utf8]{inputenc}
\usepackage{amsmath,amsfonts,amssymb}
\usepackage{xcolor}
\usepackage{listings}
\usepackage[backend=biber,style=numeric-comp, sorting=none]{biblatex}
\lstset{
  basicstyle=\ttfamily,
  columns=fullflexible,
}
\definecolor{light-gray}{gray}{0.95}
\addbibresource{ref.bib}
\begin{document}
\tableofcontents
\thispagestyle{empty}
\newpage
\section{Motivation - Why ARGES?}
Arges is the name of a cyclops in the greek mythology, which was exiled by the Titans but freed by Zeus to forge lightning bolts for his battle against his former masters. \newline
In this tradition as a useful tool, ARGES - Advanced Renormalization Group Equation Simplifier is a framework to calculate perturbative renormalization group equations for renormalizable quantum field theories, using \cite{LuoWangXiao,MV1,MV2,MV3,vev1,vev2,OsbornJack,OsbornJack2,OsbornJack3,Decoding}. Of course, other tools exist which offer an intersection of the same core features, most famously SARAH \cite{SARAH,SARAH4} and PyR@TE \cite{pyrate,pyrate2}. However, this piece of software tries to adress some of their limitations and adopts their resolution as its guiding features:
\begin{itemize}
\item it is possible to define particle numbers, representations and gauge group ranks as variables, even at all at once
\item gauge groups may be defined as totally unknown altogether
\item gauge invariants are not automatically resolved
\item for interaction terms, contractions of generation and gauge indices are completely configurable
\item disentanglement of parameter RGEs is up to the user, hence the program cannot fail while trying to resolve these
\end{itemize}
ARGES is limited to irreducible representation and RGEs in the $\overline{\mathrm{MS}}$ scheme, however, for all mass independent schemes, beta functions are universal at one loop (two loop for gauge couplings).
\newpage
\section{Prerequisites}

\newpage
\section{Defining a model}
\subsection{Gauge sector and initial Setup}
The initial step when writing a model file is to invoke the function $\mathtt{Reset[\;];}$ to initialize background variables. This command also clears the package and allows a fresh and redefinition of a model. This is particularly useful when accidentally adding fields or couplings twice. \\
A second step is the to define the variable $\mathtt{NumberOfSubgroups}$, specifying the number of gauge subgroups. The default for this variable is set to 1, but it can be set to any non-negative integer, including zero. The value assigned to $\mathtt{NumberOfSubgroups}$ cannot be a symbol.
\vspace{1em}
\begin{lstlisting}[language=mathematica,mathescape,columns=flexible,backgroundcolor=\color{light-gray}]
Reset[];
NumberOfSubgroups=1;
\end{lstlisting}
\vspace{1em}
In order to define gauge groups and their respective fields the function $\mathtt{Gauge[\dots]}$ must be invoded $\mathtt{NumberOfSubgroups}$ times. The syntax is:
\begin{lstlisting}[language=mathematica,mathescape,columns=flexible,backgroundcolor=\color{light-gray}]
Gauge[Symbol, Algebra, Rank, {Charge/Representation1, ...}];
\end{lstlisting}
The last argument is a list of the charges (in case of $\mathrm{U}(1)$ subgroups) or representations of gauge bosons. The first charge/representation will be associated with the group first declared with $\mathtt{Gauge[\dots]}$, and so on. The dimension of this list must match $\mathtt{NumberOfSubgroups}$. ARGES is agnostic regarding anti-representations, only positive numbers are valid. However, charges, representations and rank of the subgroup may be declared as a variable. \newline Currently available algebras and ranks are $\mathrm{U}(1)$ , $\mathrm{SU}(N_U)$, and $\mathrm{SO}(N_O)$ with $N_U > 1$ and $N_O > 2$. Gauge fields are assumed to be in the adjoint representation of their own subgroup and singulets in the others. Kinetic mixing is not implemented. \newline Unknown Gauge groups are allowed, but their invariants can of course not be further resolved.
\subsection{Fermionic fields}
Fermions may be defined via $\mathtt{WeylFermion}$:
\begin{lstlisting}[language=mathematica,mathescape,columns=flexible,backgroundcolor=\color{light-gray}]
WeylFermion[Symbol, Generations, {Charge/Representation1, ...}];
\end{lstlisting}
The same comments as for $\mathtt{Gauge[\dots]}$ apply here for the last argument, including the fact that the representations and charges may be defined as variables. The same is true for the number of generations.
\subsection{Scalar fields}
Scalars may be defined as real or complex, using the following declarations:
\begin{lstlisting}[language=mathematica,mathescape,columns=flexible,backgroundcolor=\color{light-gray}]
RealScalar[Symbol, {Gen1, Gen2}, {Charge/Representation1, ...}];
ComplexScalar[Symbol, {Gen1, Gen2}, {Charge/Representation1, ...}];
\end{lstlisting}
While most of the syntax matches the fermionic case, scalars are natively matrix fields and posses two generation indices. For a vector-like generation structure, one of them may be set to 1.\newline
Internally, all complex scalars are expanded into real components, using the decomposition $\Phi = \frac{1}{\sqrt{2}} \left(\mathrm{Re} \, \Phi + i \, \mathrm{Im} \,\Phi\right)$. These component fields may be referenced as $\mathtt{Re[\Phi]}$ and $\mathtt{Im[\Phi]}$ when defining interactions.
\subsection{Yukawa interactions}
Two possibilities exist to define Yukawa interactions. One is the classical approach as a matrix in the fermionic generation indices, the syntax looks like this:
\begin{lstlisting}[language=mathematica,mathescape,columns=flexible,backgroundcolor=\color{light-gray}]
YukawaYaij[Symbol, Scalar, Ferm1, Ferm2, {Contraction1, ...}, Factor];
\end{lstlisting}
The arguments two to four are the symbols of the fields involved. The Scalar entered may be real (or the real/imaginary part of complex field - using $\mathtt{Re[...]}$/$\mathtt{Im[...]}$) or complex. The hermitian conjugate of a complex field may be used as well, utilizing $\mathtt{adj[...]}$. The first fermion is considered hermitian conjugated as well, but there is no need to mark the symbol. The last argument is a prefactor which might be complex, but its usage is optional, and the default value 1. The fifth arguments is to define the contraction between the gauge indices of the fields involved for this term. This list has $\mathtt{NumberOfSubgroups}$ elements and contains functions with three arguments, where which correspond to the gauge indices of the scalar, first and second fermion, strictly in this order. The hermitian conjugate of this Yukawa term is added automatically, no further prefactor or symmetrization will be employed. There is no contraction for scalar generations for these terms, but traces and products in the fermionic indices are denoted as $\mathtt{tr[...]}$ and $\mathtt{prod[...][i,j]}$ respectively.\newline 
Furthermore, Yukawa couplings may be defined as scalars:
\begin{lstlisting}[language=mathematica,mathescape,columns=flexible,backgroundcolor=\color{light-gray}]
YukawaY[Sym, Scalar, Ferm1, Ferm2, {Contr1, ...}, GenerationContr];
\end{lstlisting}
Which is the same thing with exception of the now mandatory last argument, which is a now a function with four arguments, and defines the contraction between the first and second generation index of the scalar, and the generation indices of the first and second fermion, strictly in that order. Again, the first arguments is auto-adjoint with the need to mark it that way, and the hermitian conjugate is added without further symmetry factors.
\subsection{Scalar quartic interactions}
Scalar quartics may be defined via:
\begin{lstlisting}[language=mathematica,mathescape,columns=flexible,backgroundcolor=\color{light-gray}]
Quartic$\lambda$abcd[Sym, Sc1, Sc2, Sc3, Sc4, {Contr1, ...}, GenerationContr];
\end{lstlisting}
Where arguments two to five are symbols of the scalars involved. Again, real and complex scalars may be entered. It may be denoted that for complex fields, prefactors of the real composite fields are added automatically. The sixth element is a list of gauge index contractions, taking four arguments reflecting the four scalar fields. The generation contraction on the last position is a function of eight arguments, where the first two are for the two generation indices of the first scalar field and so on. \newline No implicit factor 4! is added, so each symmetry factor needs to be included in the last argument. However, for calculational purposes, the quartics are completely symmetrized under permutation of the scalar fields, and additional factors are added to preserve the norm. The final prefactor may be checked using the routine to output beta function, as it is explained in the next chapter.
\subsection{Vacuum expectation values}
For theories developing VEVs of scalars, these have to be declared before using:
\begin{lstlisting}[language=mathematica,mathescape,columns=flexible,backgroundcolor=\color{light-gray}]
VEV[Symbol, Scalar, {Gen1, Gen2}, {GaugeIndex1, ...}, Factor];
\end{lstlisting}
Where the first two arguments specify the symbols of the VEV and the scalar field. Since VEVs are considered to be real, real component fields need to be utilized for the scalars. However, its is of course possible to define imaginary components to develop a VEV. The third and fourth arguments are Lists containing the generation and gauge indices of the field in the usual fashion. The last argument is an additional factor for the VEV, however, factors already defined for the component fields are implicitly assumed for the VEV as well. This argument is optional and defaults to one. In general, to facilitate more complex vacuum structures, multiple VEVs with the same symbol, but other arguments or different VEVs with the same arguments may be specified, these are always assumed in an additive manner.
\newpage
\section{Output}
\subsection{General remarks}
Beta functions are in general calculated on demand, the user has full control over which function to calculate and to which loop order. The general syntax follows $\mathtt{\beta[..., Loop Order]}$. Other inputs may be the involved fields and their quantum numbers. If the loop order is set to 0, the linear combination of couplings is returned as written in the Lagrangian modulo the fields and indices specified. This is particularly useful to find out prefactors after symmetrization and from taking real and imaginary parts of complex scalars. This is also the main mechanism to disentangle the running of couplings.\newline The gauge and generation indices of the involved fields are entered as lists. For abelian gauge groups, it is necessary to insert 1 instead of the charge. If the representation, generations or rank are variables (for both the definition or the input indices), the situation is slightly more complicated, since the Mathematica kernel tries to cover all special cases when handling sums with variables, and drags them through the entire computation. This may slow the calculation severely or even break it. ARGES utilizes the global variable \$\texttt{Assumptions} (be careful if you use it as well, \texttt{Reset[]} wipes it completely) to refine these cases. If variables appear in the definition, the index 1 is always safe as input, for higher numbers, a global assumption must be appended to respecify the range of the variables. If there is an input of variables, their range as well as the fact that these are integers must be appended as well. \newline 
Gauge invariants are by default not resolved. To do so, the function $\mathtt{ComputeInvariants[\;]}$ must first be evoked, which fills the replacement rule list \texttt{subInvariants} with appropriate substitutions. It shall be denoted that for scalars, at two loop in the Yukawa betas and one loop in the quartic betas, as well as for fermions at two loop in the quartic betas, gauge terms appear which may not be expressed by Casimirs and Dynkin Indices, but rather: $ \Lambda_{ijkl} = T ^a_{ik}\;T^a_{jl}$, where $T^a_{ij}$ are generators of a gauge subgroup for one particle. However, closed formulas for any rank may not exist for all representations, as it is the case for instance for the fundamental representation of SU(N).
\subsection{Gauge coupling beta function}
For gauge couplings, the beta functions may be calculated via:
\begin{lstlisting}[language=mathematica,mathescape,columns=flexible,backgroundcolor=\color{light-gray}]
$\beta$[Gauge, LoopOrder]
$\beta$[$\alpha$[Gauge], LoopOrder]
\end{lstlisting}
Where the second version returns the beta function of and with $ \alpha(g) = \frac{g^2}{(4 \pi)^2}$. In any case, the functions still contain Casimirs and Dynkin Indices. To resolve these (when the Algebra and the representation is specified), the substitution \texttt{subInvariants} may be utilized.
\subsection{Yukawa beta function}
\begin{lstlisting}[language=mathematica,mathescape,columns=flexible,backgroundcolor=\color{light-gray}]
$\beta$[Scalar, Ferm1, Ferm2, SIdxList, F1IdxList, F2IdxList, LoopOrder]
\end{lstlisting}
The first three arguments are symbols of the scalar and the fermions involved, the fourth argument is a list of $\mathtt{NumberOfSubgroups} + 2$ elements, where the first two mark the generations of the scalar fields, and the rest its gauge indices. In the same manner, the fifth and sixth arguments are lists of $\mathtt{NumberOfSubgroups} + 1$ elements, where the first argument is the generation, and the rest gauge indices for each fermion respectively.\newline
\subsection{Scalar quartic beta function}
\begin{lstlisting}[language=mathematica,mathescape,columns=flexible,backgroundcolor=\color{light-gray}]
$\beta$[Sc1, Sc2, Sc3, Sc4, S1List, S2List, S3List, S4List, LoopOrder]
\end{lstlisting}
Following the notation before, the first four symbols mark the scalar fields, while the arguments five to eight are their respective indices, packed in lists of $\mathtt{NumberOfSubgroups} + 2$ elements, with the first two being the generation and the rest the gauge indices. \newline
\subsection{Vacuum expectation values}
The running of vacuum expectation values is obtained via:
\begin{lstlisting}[language=mathematica,mathescape,columns=flexible,backgroundcolor=\color{light-gray}]
$\beta$[vev, LoopOrder]
\end{lstlisting}
It may be denoted that these are in general gauge depended. Here, all gauge groups are assumed have the same constant $\xi$ for $R_{\xi}$ gauge fixing.
\newpage
\section{Example: SM with $N_c$ colors and third generation fermions}
In this section, we will present an example of usage featuring the Standard Model with $N_c$ colors and only considering third generation fermions. The model file looks like this (depicted in plain text without notebook specific replacements):
\begin{lstlisting}[language=mathematica,mathescape,columns=flexible,backgroundcolor=\color{light-gray}]
(* SM with Nc colors and 3rd generation fermions only *)
Needs["ARGES`"]

(* Reset ARGES every time reading in this file *)
Reset[];

(* Declare number of gauge groups first *)
NumberOfSubgroups = 3;

(* Specify gauge groups, no GUT normalization *)
Gauge[gy, U, 1, {0, 1, 1}];
Gauge[g, SU, 2, {0, 3, 1}];
Gauge[gs, SU, Nc, {0, 1, Sqr[Nc] - 1}];

(* Specify full fermion sector *)
(* we will define no Yukawas for the first 2 generations later *)
WeylFermion[Q, 3, {+1/6, 2, Nc}];
WeylFermion[L, 3, {-1/2, 2, 1}];
WeylFermion[D, 3, {-1/3, 1, Nc}];
WeylFermion[U, 3, {+2/3, 1, Nc}];
WeylFermion[E, 3, {-1, 1, 1}];

(* Scalar Field with VEV *)
ComplexScalar[H, {1, 1}, {+1/2, 2, 1}];
VEV[v, Re[H], {1, 1}, {1, 2, 1}, 1];

(* Yukawa sector - third generations only *)
YukawaY[ yb, H, Q, D, 
 {1 &, KroneckerDelta[#1, #2] &, KroneckerDelta[#2, #3] &}, 
 (KroneckerDelta[#3, 3] KroneckerDelta[#4, 3])& ];
 
YukawaY[ ytau, H, L, E, 
 {1 &, KroneckerDelta[#1, #2] &, 1 &}, 
 (KroneckerDelta[#3, 3] KroneckerDelta[#4, 3])& ];
 
YukawaY[ yt, adj[H], Q, U, 
 {1 &, Eps[#1, #2] &, KroneckerDelta[#2, #3] &},
 (KroneckerDelta[#3, 3] KroneckerDelta[#4, 3])& ];

(* Quartic coupling ~ -1/2 lamda |phi|^4 *)
Quartic\[Lambda]abcd[\[Lambda], adj[H], H, adj[H], H, 
 {1&, (KroneckerDelta[#1, #2] KroneckerDelta[#3, #4])&, 1&}, 
 (KroneckerDelta[#2, #3] KroneckerDelta[#1, #4] 
  KroneckerDelta[#5, #8] KroneckerDelta[#6, #7])/2 & ];

(* Compute gauge invariants already *)
ComputeInvariants[];
\end{lstlisting}
This file is included in the bundle as $\mathtt{ARGES/models/SM\_Nc\_3rd.m}$. It shall be denoted that $\mathtt{Eps}$ and $\mathtt{Sqr}$ are keywords defined by ARGES for convenience, while $\mathtt{KroneckerDelta}$ is part of the Wolfram Language. It is less cumbersome to utilize the support for anonymous ('pure') functions as done in the example.
\begin{lstlisting}[language=mathematica,mathescape,columns=flexible,backgroundcolor=\color{light-gray}]
(* non-anonymous definition *)
f[x1_, x2_, ...] := g[x1, x2, ...];

(* anonymous definition using pure functions *)
f := g[#1, #2, ...] & ;
f := Function[{x1, x2, ...}, g[x1, x2, ...]]; 
\end{lstlisting}
Now the model can be called from a batch file or interactively from a notebook:
\begin{lstlisting}[language=mathematica,mathescape,columns=flexible,backgroundcolor=\color{light-gray}]
In[1]:= Get["SM_Nc_3rd.m", Path -> {"~/ARGES/models/"}]
        
        (* real matrix elements, substitute gauge invariants *)        
In[2]:= subs = Join[subInvariants, 
        {adj[yt] -> yt, adj[yb] -> yb, adj[ytau] -> ytau}];

        (* gauge couplings beta functions - here just gs for brevity *)        
In[3]:= Power[4 \[Pi], 2] \[Beta][gs, 1] //. subs // Expand
Out[3]= 4 gs^3 - (11 gs^3 Nc)/3

In[4]:= Power[4 \[Pi], 4] \[Beta][gs, 2] //. subs // Expand
Out[4]= (9 g^2 gs^3)/2 + (11 gs^3 gy^2)/6 - (6 gs^5)/Nc + 26 gs^5 Nc - 
        (34 gs^5 Nc^2)/3 + (2 gs^3 yb^2)/(-1 + Nc^2) - 
        (2 gs^3 Nc^2 yb^2)/(-1 + Nc^2) + (2 gs^3 yt^2)/(-1 + Nc^2) - 
        (2 gs^3 Nc^2 yt^2)/(-1 + Nc^2)
        
        (* Yukawa yt coupling: get symmetry factors first *)
In[5]:=  \[Beta][H, Q, U, {1, 1, 1, 1, 1}, {3, 1, 2, 1}, 
         {3, 1, 1, 1}, 0] //. subs // Expand
Out[5]= yt

In[6]:= Power[4 \[Pi], 2] \[Beta][H, Q, U, {1, 1, 1, 1, 1}, 
        {3, 1, 2, 1}, {3, 1, 1, 1}, 1] //. subs // Expand
Out[6]= -((9 g^2 yt)/4) - (17 gy^2 yt)/12 + (3 gs^2 yt)/Nc - 
        3 gs^2 Nc yt - (3 yb^2 yt)/2 + Nc yb^2 yt + (3 yt^3)/2 + 
        Nc yt^3 + yt ytau^2
        
In[7]:= Power[4 \[Pi], 4] \[Beta][H, Q, U, {1, 1, 1, 1, 1}, 
        {3, 1, 2, 1}, {3, 1, 1, 1}, 2] //. subs // Expand
Out[7]= -8 g^4 yt + (53 gs^4 yt)/3 - 3/4 g^2 gy^2 yt + (61 gy^4 yt)/24 - 
        (3 gs^4 yt)/(4 Nc^2) - (27 g^2 gs^2 yt)/(8 Nc) - (10 gs^4 yt)/Nc
        - (19 gs^2 gy^2 yt)/(24 Nc) + 3/4 g^4 Nc yt + 27/8 g^2 gs^2 Nc yt
        + 10 gs^4 Nc yt + 19/24 gs^2 gy^2 Nc yt + 319/324 gy^4 Nc yt - 
        203/12 gs^4 Nc^2 yt + 9/16 g^2 yb^2 yt - 5/2 gs^2 yb^2 yt - 
        43/48 gy^2 yb^2 yt + (6 gs^2 yb^2 yt)/Nc + 15/8 g^2 Nc yb^2 yt -
        6 gs^2 Nc yb^2 yt + 25/72 gy^2 Nc yb^2 yt + 5/2 gs^2 Nc^2 yb^2 yt
        + (11 yb^4 yt)/4 - Nc yb^4 yt + (135 g^2 yt^3)/16 - 
        (5 gs^2 yt^3)/2 + (223 gy^2 yt^3)/48 - (6 gs^2 yt^3)/Nc + 
        15/8 g^2 Nc yt^3 + 6 gs^2 Nc yt^3 + 85/72 gy^2 Nc yt^3 + 
        5/2 gs^2 Nc^2 yt^3 - (5 yb^2 yt^3)/4 - 1/2 Nc yb^2 yt^3 + 
        (3 yt^5)/2 - (9 Nc yt^5)/2 + 15/8 g^2 yt ytau^2 + 
        25/8 gy^2 yt ytau^2 + 5/4 yb^2 yt ytau^2 - (9 yt^3 ytau^2)/4 - 
        (9 yt ytau^4)/4 - 6 yt^3 \[Lambda] + (3 yt \[Lambda]^2)/2
\end{lstlisting}
\newpage
\printbibliography
\end{document}